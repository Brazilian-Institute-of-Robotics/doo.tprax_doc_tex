\begin{agradecimentos}
	Agradecemos primeiramente ao Brazilian Institute of Robotics (BIR), através da figura de Leonardo Nardy e Marco Reis, pela oportunidade e confiança em nossa equipe na realização deste projeto; e ainda mais diretamente ao professor Marco por todo suporte que nos foi oferecido como orientador ao longo desse TheoPrax. Ao professor João da Hora,  por da mesma forma ter acompanhado nossas atividades e pela sua cooperação em nos conciliar com a metodologia TheoPrax, mesmo com as especificidades trazidas por esta Equipe. Aos coordenadores de curso Guilherme Souza e Taniel Franklin, pela agilidade e disponibilidade, mesmo em final de semestre, em possibilitarem a execução tanto da banca quanto a pré-banca. Por fim, a Equipe agradece ao SENAI Cimatec por toda sua infraestrutura, seja pelo Laboratório TheoPrax, que nos serviu como base para a montagem do robô, ou pela cooperação com seus demais setores que nos concedeu parte dos materiais necessários para a confecção dos nossos protótipos, a vocês nosso muito obrigado!
	
	Eu, Élisson Oliveira, agradeço à minha família, pelo apoio durante toda essa jornada. Agradeço pela compreensão de minhas duas mães, Dona Zina e Sandra, durante os momentos de minha ausência. Agradeço ao meu pai, Sr. Roberto, sem o qual eu não teria conquistado nada do que conquistei até aqui. Agradeço à Alana, pela parceria, incentivo e conforto nos momentos difíceis. Por fim, agradeço à todos os meus amigos, os mais próximos e os que estão mais distantes, por todo o aprendizado e companhia durante esse percurso.
	
	Eu, Iure Pinheiro, gostaria de agradecer a toda minha família pelo apoio durante todo o processo, em especial ao meus pais Romilda e Antônio que sempre me incentivaram na vida acadêmica e que sem eles eu não conseguiria ter chegado nesse momento. Agradeço à Tais por nunca ter me deixado desanimar durante os momentos mais difíceis e por ser um exemplo de determinação para mim. Gostaria de agradecer também à Mateus pelos conselhos e encaminhamentos no decorrer do curso e a Vitor pelas conversas que foram  necessárias para retirar o stress diário. Agradeço também ao meu primo Davi e meu amigo de longa data Leandro pelas palavras de motivação e conforto. Por último e não menos importante, agradeço a todos os meus amigos que não foram citados mas que contribuíram mesmo que indiretamente para a conclusão deste ciclo.
	
	Eu, Mateus Meneses, agradeço especialmente a meus pais Arnaldo e Maria por todo apoio emocional e incentivo ao longo dessa longa jornada. Agradeço também aos meus irmãos e irmã por oferecer um ambiente familiar tão sólido e motivador, o que foi de extrema importância para lidar com os momentos difíceis ao longo da graduação. Agradeço ao apoio desde a infância dos meus amigos Renato e Walter e suas respectivas famílias. Agradecimentos a Iure e Vitor pelos momentos de descontração e as conversas aleatórias nos corredores do SENAI CIMATEC. A todos os colegas do BIR, muito obrigado por estarem sempre dispostos a compartilharem o conhecimento técnico, profissional e pessoal de vocês. Meus agradecimentos também a Fred por compartilhar as experiências e desafios do TheoPrax. Por fim, agradeço a todos aos meus amigos que não foram citados diretamente aqui mas que de alguma forma me ajudaram durante os anos de graduação.
\end{agradecimentos}