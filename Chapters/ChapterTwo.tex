\chapter{Fundamentação Teórica}
\label{chap:fundteor}

\begin{flushright}

   \begin{list}{}{
      \setlength{\leftmargin}{4.5cm}
      \setlength{\rightmargin}{0cm}
      \setlength{\labelwidth}{0pt}
      \setlength{\labelsep}{\leftmargin}}
      \item Quanto maior for a rapidez de transformação de uma
      sociedade, mais temporárias são as necessidades
      individuais. Essas flutuaçõess tornam ainda mais acelerado
      o senso de turbilh da sociedade.

      \begin{list}{}{
      \setlength{\leftmargin}{0cm}
      \setlength{\rightmargin}{0cm}
      \setlength{\labelwidth}{0pt}
      \setlength{\labelsep}{\leftmargin}}
      \item (Alvin Toffler)
      \end{list}
   \end{list}
\end{flushright}

\begin{flushright}
  Quanto maior for a rapidez de transformação de uma \\
  sociedade, mais temporárias são as necessidades \\
  individuais. Essas flutuações tornam ainda mais \\
  acelerado o senso de turbilhão da sociedade. \\
  \ \\
  (Alvin Toffler)
\end{flushright}

%--------- NEW SECTION ----------------------
\section{Micromouse}
\label{sec:Micromouse}


%--------- NEW SECTION ----------------------
\section{Robotics Frameworks}
\label{sec:robotic_frameworks}



%--------- NEW SECTION ----------------------
\section{Estudo do estado da arte}
\label{sec:sota}
 \hspace{0.5cm} A competição Micromouse é um concurso anual na qual estudantes do mundo todo desenvolvem pequenos robôs autônomos, denominado Micromouse, postos a correr dentro de um labirinto. O autônomo que mais rápido chegar ao seu centro é o vencedor da competição.

 \hspace{0.5cm}  Ainda em 1977, a \textit{IEEE Spectrum Magazine} trouxe pela primeira vez o conceito de um robô autônomo para resolução de labirintos. Sua primeira competição logo foi realizada, em junho de 1979, na primeira \textit{IEEE Amazing Micromouse Maze Contest} organizada em Nova York. Rapidamente, o conceito da competição se espalhou, e até o começo da década de 90, vários clubes voltados para Micromouse surgiam em escolas e universidades do mundo todo. [From: The inception of Chedda]


%--------- NEW SECTION ----------------------
\section{Assunto 1}
\label{sec:ass1}
flkjasdlkfjasdlkfjs

%--------- NEW SECTION ----------------------
\section{Assunto 2}
\label{sec:ass2}
flkjasdlkfjasdlkfjs

