\chapter{Conclusão}
\label{chap:conc}

Neste trabalho foi apresentada uma plataforma open source para ensino de robótica e introdução à inteligência artificial. Foram confeccionados dois protótipos, capazes de percorrer e solucionar um labirinto, utilizando algoritmos de busca.

O robô foi desenvolvido utilizando o framework ROS e o ambiente de simulação Gazebo, dando a possibilidade de se trabalhar e experimentar com os robôs reais e com o ambiente simulado.

Cerca de 90\% dos pacotes e 75\% das stacks foram completamente desenvolvidas pela equipe do projeto. Procurou-se ao máximo evitar a “reinvenção da roda”, utilizando estratégias do ramo da robótica que já são bem consolidadas na comunidade.

Foram desenvolvidos guias de montagem, modelo 3D, desenhos técnicos, guias de configuração de software e documentação independente dos pacotes. Tudo isso com o intuito de auxiliar o usuário durante o seu processo de aprendizagem.

Foram utilizadas boas práticas de desenvolvimento, além de padrões utilizados em outros projetos de robótica. Dessa forma, buscou-se obter um sistema que possa ser bem recebido tanto por iniciantes quanto por pessoas já familiarizadas com a área.

É possível perceber que a plataforma consegue extrapolar o seu objetivo inicial: além de ser utilizada como um robô solucionador de labirintos, nela podem ser aplicadas técnicas de \gls*{slam}, técnicas diferentes de controle, além da utilização de técnicas de navegação de robôs autônomos como \textit{State Machine} e \textit{Behavior Tree}. 

A realização desse projeto foi bastante enriquecedora, proporcionando aos discentes grande aprendizado. Todos os participantes conseguiram aprimorar suas habilidades, tanto na área profissional, quanto na área pessoal.

%--------- NEW SECTION ----------------------
\section{Trabalhos futuros}
\label{sec:trabfut}
Para melhorar a experiência do usuário com o Doogie Mouse pretende-se no futuro realizar as seguintes ações:
\begin{itemize}
	\item Elaborar uma Wiki no GitHub reunindo toda documentação do robô;
	\item Confeccionar um labirinto para testes do protótipo físico;
	\item Desenvolver três estratégias de resolução de labirinto para servirem como exemplo base de nova implementações;
	\item Realizar a fusão de dados de Odometria e IMU para oferecer uma maior precisão de movimento do robô.
\end{itemize}
 

